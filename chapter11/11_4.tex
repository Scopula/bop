\documentclass{article}
\usepackage[left=8em,right=8em]{geometry}
\usepackage{amsmath,amsthm,amssymb}
\usepackage{lineno}
\usepackage{relsize}
\usepackage{graphicx}
\usepackage{hyperref}

\hypersetup{
    colorlinks=true,
    linkcolor=blue,
}
%\linenumbers
\renewcommand*{\proofname}{Proof}
\date{}
\author{}
\begin{document}
\centerline{\textbf{Exercises for Section 11.4}}
$ $\newline
\textbf{Ex 1.}\\
Partitions on $A=\{a,b\}$
\begin{enumerate}
\item \{\{a, b\}\}
\item \{\{a\}, \{b\}\}
\end{enumerate}
\textbf{Ex 2.}\\
Partitions on $A=\{a,b,c\}$
\begin{enumerate}
\item \{\{a, b, c\}\}
\item \{\{a\}, \{b, c\}\}
\item \{\{b\}, \{a, c\}\}
\item \{\{c\}, \{a, b\}\}
\item \{\{a\}, \{b\}, \{c\}\}
\end{enumerate}
\textbf{Ex 3.}\\
The equivalence relation $\equiv \pmod{4}$ partitions $\mathbb{Z}$ into\\ $\{[0], [1], [2], [3]\}=\{\{...,-4,0,4,...\},\{...,-3,1,5,...\},\{...,-2,2,6,...\}, \{...,-1,3,7,...\},\}$.\\

\noindent \textbf{Ex 4. Proposition:} Suppose $P$ is a partition of set $A$, $R$ is a relation on $A$ such that $(x, y) \in R$ if and only if $x, y \in X$ for some $X \in P$. Then $R$ is an equivalence relation on $A$.
\begin{proof}
$ $\newline
First we show that $R$ is reflexive. Suppose $x \in A$. Because $P$ is a partition of $A$, we have that $x \in X$ for some $X \in P$. By definition of $R$, it follows that $(x, x) \in R$ and consequently $R$ is reflexive.\\

\noindent Next we show that $R$ is symmetric. Suppose $(x, y) \in R$. So $x, y \in X$ for some $X \in P$. From this it directly follows that $(y, x) \in R$, thus $R$ is symmetric.\\

\noindent Finally we show that $R$ is transitive. Suppose $(a, b), (b,c) \in R$. Then $a,b \in X_1$ and $b, c \in X_2$ for some $X_1, X_2 \in P$. Then the definition of $P$ (two different subsets of P have no elements in common) combined with the fact that $b \in X_1$ and $b \in X_2$, implies that $X_1=X_2$. So $a, b, c \in X_1$, implies that $(a, c) \in R$. Thus $R$ is transitive.\\

\end{proof}

\newpage

\noindent \textbf{Ex 4. Proposition:} Suppose $P$ is a partition of set $A$, $R$ is a relation on $A$ such that $(x, y) \in R$ if and only if $x, y \in X$ for some $X \in P$. Then $P$ is the set of equivalence classes of $R$.

\begin{proof}
$ $\newline

\noindent Skipped for now. There is a non-cogent proof \href{https://www.slader.com/textbook/9780989472111-book-of-proof-2nd-edition/188/exercises/4/}{here}.
%\noindent Let $EC = \{[a] : a \in A \}$ be the set of equivalence classes of $R$. We want to show that $EC=P$. To do so we need to show that the union of sets $[a]$ equals $A$ and that if $[a]\neq [b]$, then $[a] \cap [b] = \emptyset$.

%Is this a reasonable strategy? https://imgur.com/a/ShjkDTq If I show EC is a partition on A, it must be the case that EC=P (because each equivalence relation has unique EC and P).
\end{proof}

\noindent \textbf{Ex 5.} $R$ is the equivalence relation $\equiv \pmod{2}$.\\

\noindent \textbf{Ex 6.} $R$ is the equivalence relation defined as: $(x, y) \in R$ if and only if $|x|=|y|$.\\
\end{document}

