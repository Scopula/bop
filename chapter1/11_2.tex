\documentclass{article}
\usepackage[left=8em,right=8em]{geometry}
\usepackage{amsmath,amsthm,amssymb}
\usepackage{lineno}
\usepackage{relsize}
\usepackage{graphicx}
%\linenumbers
\renewcommand*{\proofname}{Proof}
\date{}
\author{}
\begin{document}
\centerline{\textbf{Exercises for Section 11.2}}
$ $\newline
\textbf{Ex 1.} $R$ is reflexive, symmetric, and transitive.\\
$ $\newline
\textbf{Ex 2.} \\
$R$ is not reflexive because $(a,a) \not\in R$.\\
$R$ is not symmetric because $(a, b) \in R$, but $(b,a) \not\in R$.\\
$R$ is transitive.\\
$ $\newline
\textbf{Ex 3.} \\
$R$ is not reflexive because $(a,a) \not\in R$.\\
$R$ is not symmetric because $(a,b) \in R$, but $(b,a) \not\in R$.\\
$R$ is not transitive because $(c,b) \in R $ and $(b,c) \in R$, but $(c,c) \not\in R$.\\
$ $\newline
\textbf{Ex 4.} $R$ is reflexive, symmetric, and transitive.\\
$ $\newline
\textbf{Ex 5.}\\
$R$ is not reflexive because $(1,1) \not\in R$.\\
$R$ is symmetric and transitive.\\
$ $\newline
\textbf{Ex 6.}\\
This is the equality relation on $\mathbb{Z}$.\\
It is reflexive, symmetric and transitive.\\
$ $\newline
\textbf{Ex 7.}\\
The $16$ different relations on $A=\{a,b\}$:\\
\begin{itemize}
  \item 1 - $\emptyset$. This relation is symmetric and transitive, but not reflexive.
  \item 1 - $\{(a, b), (b,a)\}$. This relation is symmetric, but not reflexive and not transitive. 
  \item 2 - $\{(a,a), (b,b), (a, b), (b, a)\}, \{(a,a), (b,b)\}$. These relations are reflexive, symmetric and transitive.
  \item 2 - $\{(a,a)\}, \{(b, b)\}$. These relations are symmetric and transitive, but not reflexive.
  \item 2 - $\{(a, b)\}, \{(b,a)\}$. These relations are transitive, but not reflexive and not symmetric.
  \item 4 - $\{(a,a),(a, b)\}, \{(a,a), (b,a)\}, \{(b,b),(a,b)\}, \{(b,b),(b,a)\}$ . These relations are transitive, but not reflexive and not symmetric.  
  \item 2 - $\{(a,a), (b, b), (a, b)\},\{(a,a), (b, b), (b, a)\}$. These relations are reflexive and transitive, but not symmetric.
  \item 2 - $\{(a,a), (a, b), (b, a)\},\{(b,b), (a, b), (b, a)\}$. These relations are symmetric, but not reflexive and not transitive.
\end{itemize}
\newpage
\noindent\textbf{Ex 8.}\\
$R=\{(x, y) \in \mathbb{Z}\times\mathbb{Z} : |x-y|<1 \}$. This relation is the equality relation on $\mathbb{Z}$. The relation is reflexive, symmetric, and transitive.\\
$ $\newline
\textbf{Ex 9.}\\
$R=\{(x, y) \in \mathbb{Z}\times\mathbb{Z} : 2|x-y \}$. This relation is the $\equiv \pmod{2}$ relation on $\mathbb{Z}$. It is reflexive, symmetric and transitive.\\
$ $\newline
\textbf{Ex 10.}\\
$R$ is symmetric and transitive, but not reflexive. Suppose $x \in A$. $R$ is not reflexive because $(x,x) \not\in R$.\\
$ $\newline
\textbf{Ex 11.}\\
$R$ is reflexive, symmetric, and transitive.\\
$ $\newline
\textbf{Ex 12. Proposition:} The $|$ (division) relation on $\mathbb{Z}$ is reflexive and transitive.
\begin{proof}
$ $\newline
The division relation is clearly reflexive as every integer is divisable by itself.\\
$ $\newline
We now use the direct approach to show that the relation is transitive. Suppose $x|y$ and $y|z$ where $x,y,z \in \mathbb{Z}$. So we have that $y=xa$ and $z=yb$ for some integers $a,b$, by defintion of divisibility. Plugging $y$ in $z=yb$, we get $z=x(ab)$. Thus $x|z$.\\
\end{proof}
\noindent \textbf{Ex 13. Proposition:} If $R=\{(x, y) \in \mathbb{R} \times \mathbb{R} : x - y \in \mathbb{Z}\}$, then $R$ is reflexive, symmetric and transitive.
\begin{proof}
$ $\newline
First we show that the relation is reflexive. Note that $x-x=0 \in \mathbb{Z}$ for all $x \in R$. Thus the relation is reflexive.\\
$ $\newline
Next we show that the relation is symmetric. Let $(x, y) \in R$. Then $x-y=z$ and $z \in \mathbb{Z}$, by definition of the relation. Multiply both sides by $-1$, we get $y-x=-z$. Because $z$ is an integer, it follows that $-z$ is an integer. Therefore $(y,x) \in R$ and consequently $R$ is symmetric.\\
$ $\newline
Finally we show that the relation is transitive. Let $(x, y) \in R$ and $(y, z) \in R$. Then $x-y=a \in \mathbb{Z}$ and $y-z=b \in \mathbb{Z}$. Plugging the second equation into the first, we get $x-y=a \leftrightarrow x - (b+z) = a \leftrightarrow x-z=a+b$. The sum of two integers is an integer, so $a+b$ is an integer and thus $(x, z) \in R$. Hence the relation is transitive.\\
\end{proof}
\newpage
\noindent \textbf{Ex 14. Proposition:} If $R$ is a symmetric and transitive relation on set $A$, and there is an element $a \in A$ for which $(a,x) \in R$ for every $x \in A$, then $R$ is reflexive.
\begin{proof}
(Direct.)\\
Because $(a,x) \in R$ and $R$ is symmetric, it follows that $(x,a) \in R$. Then by the definition of transitive property, since $(x,a) \in R$ and $(a,x) \in R$, it must be the case that $(x,x) \in R$. Thus $R$ is reflexive.\\
\end{proof}
\noindent \textbf{Ex 15.} The proposition is false. Let $|A|=1$ and $R=\emptyset$ be a relation on $A$. Then $R$ is symmetric and transitive, but not reflexive.\\
$ $\newline
\noindent \textbf{Ex 16. Proposition:}	If $R$ is the relation $R=\{ (x, y) \in \mathbb{Z}\times\mathbb{Z} : x^2 \equiv y^2 \pmod{4} \}$, then it is reflexive, symmetric and transitive.\\
\begin{proof}
(Direct.)

\noindent First we show that the relation is reflexive. Suppose $z \in \mathbb{Z}$. Then $z^2-z^2=0$ and $4|0$ implies that $z^2 \equiv z^2 \pmod{4}$. Thus $(z,z) \in R$.\\

\noindent Next we show the symmetric property. Suppose $(a,b) \in R$. Then by definition of modular congruence, it follows that $4|a^2-b^2$. Thus $a^2-b^2=4x$ for some integer $x$. Then we multiply both sides by $-1$, so $a^2-b^2=4x \leftrightarrow (-1)(a^2-b^2)=4x(-1) \leftrightarrow (b^2-a^2)=4(-x)$. Thus $4|b^2-a^2$ and consequently $(b,a) \in R$. Thus the relation is symmetric.\\

\noindent Finally, we show the transitive property. Suppose $(a,b) \ in R$ and $(b,c) \in R$. Then it follows that $a^2-b^2=4i$ and $b^2-c^2=4j$ for some integers $i, j$. Adding the second equation to the first, we get $a^2-b^2+(b^2-c^2)=4i+4j \leftrightarrow a^2-c^2=4(i+j)$. Thus $4|a^2-c^2$ and consequently $(a,c) \in R$. Thus the relation is transitive.\\
\end{proof}

$ $\newline
\noindent \textbf{Ex 17.} The relation is reflexive and symmetric, but not transitive. It is not transitive because $(1,2) \in R$ and $(2,3) \in R$, but $(1,3) \not\in R$.\\
$ $\newline
\noindent \textbf{Ex 18.}
\noindent A reflexive, symmetric, but not transitive relation:\\
Let $R= \{(a,b) \in \mathbb{Z}^2 : a = b\} \cup \{(1,2), (2,1), (2,3), (3,2)\}$ be a relation on $\mathbb{Z}$.\\
$ $\newline
A reflexive, but not symmetric and not transitive relation:\\
Let $R= \{(a,b) \in \mathbb{Z}^2 : a = b\} \cup \{(0,1), (0,2)\}$ be a relation on $\mathbb{Z}$.\\
$ $\newline
A symmetric and transitive, but not reflexive relation:\\
%Let $R=\{(a,a)\}$ be a relation on $A=\{a,b\}$.\\
Let $R = \{(a,b) \in \mathbb{Z}^2 : a = 1, a = b\} = \{ (1,1) \}$ be a relation on $\mathbb{Z}$.
\end{document}

