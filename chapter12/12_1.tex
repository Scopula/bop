\documentclass{article}
\usepackage[left=8em,right=8em]{geometry}
\usepackage{amsmath,amsthm,amssymb}
\usepackage{lineno}
\usepackage{relsize}
\usepackage{graphicx}
\usepackage{hyperref}

\hypersetup{
    colorlinks=true,
    linkcolor=blue,
}
%\linenumbers
\renewcommand*{\proofname}{Proof}
\date{}
\author{}
\begin{document}
\centerline{\textbf{Exercises for section 12.1}}
$ $\newline
\textbf{Ex 1.} Domain: $A=\{0,1,2,3,4\}$, range: $\{2,3,4\}$, $f(2)=4$, and $f(1)=3$.\\

\noindent\textbf{Ex 2.} Domain: $A=\{a,b,c,d\}$, range: $\{2,3,4,5\}$, $f(b)=3$, and $f(d)=5$.\\

\noindent\textbf{Ex 3.} Four different $f$ functions:
\begin{enumerate}
\item $f = \{(a,0), (b, 0)\}$
\item $f = \{(a,0), (b, 1)\}$
\item $f = \{(a,1), (b, 0)\}$
\item $f = \{(a,1), (b, 1)\}$
\end{enumerate}

\noindent\textbf{Ex 4.} Eight different $f$ functions:
\begin{enumerate}
\item $f = \{(a,0), (b,0), (c,0)\}$
\item $f = \{(a,0), (b,0), (c,1)\}$
\item $f = \{(a,0), (b,1), (c,0)\}$
\item $f = \{(a,1), (b,0), (c,0)\}$
\item $f = \{(a,1), (b,1), (c,0)\}$
\item $f = \{(a,1), (b,0), (c,1)\}$
\item $f = \{(a,0), (b,1), (c,1)\}$
\item $f = \{(a,1), (b,1), (c,1)\}$
\end{enumerate}

\noindent\textbf{Ex 5.} $R = \{(a,d), (b,d),(c,d), (d,d), (d,e)\}$. $R$ is not a function because $(d,d), (d,e) \in R$.\\

\noindent\textbf{Ex 6.} Domain and codomain are both $\mathbb{Z}$. Note that $4x+5=4(x+1)+1$, because domain is $\mathbb{Z}$, the range is $\{4(x+1)+1 : x \mathbb{Z} \} = \{ ...,-7,-3,1,5,9,... \}$. Finally, $f(10)=45$.\\

\noindent\textbf{Ex 7.} Yes, $f$'s domain is $\mathbb{Z}$, range is subset of the codomain, and for each $x \in \mathbb{Z}$, $f(x)$ is uniquely defined.\\

\noindent\textbf{Ex 8.} No, for $f$ to be a function it must be the case that $f(x)$ is uniquely defined for all $x \in \mathbb{Z}$. Observe the counter-example: $f(0)$ is not defined.\\

\noindent\textbf{Ex 9.} No, for $f$ to be a function it must be the case that $f(x)$ is uniquely defined for all $x \in \mathbb{Z}$. Observe the counter-example: $f(-1)$ is not defined.\\

\noindent\textbf{Ex 10.} Yes, $f$'s domain and codmain are $\mathbb{R}$, and for each $x \in \mathbb{R}$, $f(x)$ is uniquely defined.\\

\noindent\textbf{Ex 11.} Yes. The domain is is the powerset of $\mathbb{Z}_5$. The range is $\{0,1,2,3,4, 5 \}$.\\

\noindent\textbf{Ex 12.} Yes. The domain is $\mathbb{R}^2$, the codomain is $\mathbb{R}^3$, and the range is $\{(3y, 2x, x+y) : x,y \in \mathbb{R} \}$.

\end{document}

