\documentclass{article}
\usepackage[left=8em,right=8em]{geometry}
\usepackage{amsmath,amsthm,amssymb}
\usepackage{lineno}
\usepackage{relsize}
\usepackage{graphicx}
\usepackage{hyperref}
\usepackage{parskip}

\hypersetup{
    colorlinks=true,
    linkcolor=blue,
}
%\linenumbers
\renewcommand*{\proofname}{Proof}
\date{}
\author{}
\begin{document}
\centerline{\textbf{Exercises for section 12.4}}

\textbf{Ex 1.}

$g \circ f: A \rightarrow C$ defined as $g \circ f = \{(5, 1), (6, 1), (8,1)\}$

\textbf{Ex 2.}

$g \circ f: A \rightarrow C$ defined as $g \circ f = \{(1,1),(2,1),(3,3),(4,1)\}$

\textbf{Ex 3.}

$g \circ f: A \rightarrow A$ defined as $g \circ f = \{(1,1), (2,1), (3,3)\}$

$f \circ g: A \rightarrow A$ defined as $f \circ g = \{(1,1), (2,2), (3,2)\}$

\textbf{Ex 4.}

$g \circ f: A \rightarrow A$ defined as $g \circ f = \{(a,c),(b,c),(c,c)\}$

$f \circ g: A \rightarrow A$ defined as $f \circ g = \{(a,c),(b,c),(c,c)\}$

\textbf{Ex 5.}

$g \circ f: \mathbb{R} \rightarrow \mathbb{R}$ defined as $g \circ f(x)=g(f(x))=g(\sqrt[3]{x+1})= \left(\sqrt[3]{x+1}\right)^3=x+1$

$f \circ g: \mathbb{R} \rightarrow \mathbb{R}$ defined as $f \circ g(x) = f(g(x))=f(x^3)=\sqrt[3]{x^3+1}$

\textbf{Ex 6.}

$g \circ f: \mathbb{R} \rightarrow \mathbb{R}$ defined as $g \circ f(x)=g(\dfrac{1}{x^2+1})= \dfrac{3}{x^2+1}+2$

$f \circ g: \mathbb{R} \rightarrow \mathbb{R}$ defined as $f \circ g(x) = f(g(x))=f(3x+2)=\dfrac{1}{\left(3x+2\right)^2+1}$

\textbf{Ex 7.}

$g \circ f: \mathbb{Z} \times \mathbb{Z} \rightarrow \mathbb{Z} \times \mathbb{Z}$ defined as $g \circ f(m, n)=g(f(m, n))=g(mn, m^2)= (mn+1, mn+m^2)$

$f \circ g: \mathbb{Z} \times \mathbb{Z} \rightarrow \mathbb{Z} \times \mathbb{Z}$ defined as $f \circ g(m, n) = f(g(m, n))=f(m+1, m+n)=((m+1)(m+n), (m+1)^2)$

\textbf{Ex 8.}

$g \circ f: \mathbb{Z} \times \mathbb{Z} \rightarrow \mathbb{Z} \times \mathbb{Z}$ defined as $g \circ f(m, n)=g(f(m, n))=g(3m-4n, 2m+n)= (5(3m-4n)+(2m+n),3m-4n)$

$f \circ g: \mathbb{Z} \times \mathbb{Z} \rightarrow \mathbb{Z} \times \mathbb{Z}$ defined as $f \circ g(m, n) = f(g(m, n))=f(5m+n, m)=(3(5m+n)-4m, 2(5m+n)+m)$

\textbf{Ex 9.}

$g \circ f: \mathbb{Z} \times \mathbb{Z} \rightarrow \mathbb{Z} \times \mathbb{Z}$ defined as $g \circ f(m, n)=g(f(m, n))=g(m+n)= (m+n, m+n)$

$f \circ g: \mathbb{Z} \rightarrow \mathbb{Z}$ defined as $f \circ g(m) = f(g(m))=f(m, m)=m+m$

\textbf{Ex 10.}

$f \circ f: \mathbb{R}^2 \rightarrow \mathbb{R}^2$ defined as $f \circ f(x,y) = f(f(x,y))=f(xy, x3)=(x^4y, x^3y^3)$

\end{document}